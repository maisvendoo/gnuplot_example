\documentclass[12pt]{article}

% Подключаем всяко-разное, задаем кодировку, язык и прочие параметры по вкусу
\usepackage[OT1,T2A]{fontenc}
\usepackage[utf8]{inputenc}
\usepackage[english,russian]{babel}
\usepackage{amsmath,amssymb,amsfonts,textcomp,latexsym,pb-diagram,amsopn}
\usepackage{cite,enumerate,float,indentfirst}
\usepackage{graphicx,xcolor}

% Порядку для задаем размер полей страницы, дальше это нам пригодится
\usepackage[left=2cm, right=2cm, top=2cm, bottom=2cm]{geometry}

% Включаем Gnuplottex
\usepackage{gnuplottex}

\begin{document}

\section{Построение графиков Gnuplot в документе \LaTeX}

\begin{figure}[h]
 \centering
 \begin{gnuplot}
  set terminal epslatex color size 12cm,15cm  
  set xzeroaxis lt -1
  set yzeroaxis lt -1  
  set style line 1 lt 1 lw 4 lc rgb '#4682b4' pt -1  
  set style line 2 lt 1 lw 4 lc rgb '#aa0000' pt -1  
  set grid ytics lc rgb '#555555' lw 1 lt 0
  set grid xtics lc rgb '#555555' lw 1 lt 0  
  set xrange [-3:3]  
  plot x**3 title '$y = x^3$' ls 1, \  
       x**4 title '$y = x^4$' ls 2
 \end{gnuplot}
 \caption{Пример построения двухмерных графиков}
\end{figure}

\begin{figure}[h]
 \centering
 \begin{gnuplot}
  set terminal epslatex color size 12cm,12cm
  splot x**2 + y**3 with pm3d title '$z = x^2 + y^3$'
 \end{gnuplot}
 \caption{Трехмерный график сеткой}
\end{figure}

\begin{figure}
  \centering
  \begin{gnuplot}
   set terminal epslatex color size 17cm,8cm
   set xzeroaxis lt -1
   set yzeroaxis lt -1
   set xrange [0:20]
   set style line 1 lt 1 lw 4 lc rgb '#4682b4' pt -1 
   set style line 2 lt 1 lw 4 lc rgb '#ee0000' pt -1 
   set style line 3 lt 1 lw 4 lc rgb '#008800' pt -1
   set style line 4 lt 1 lw 4 lc rgb '#888800' pt -1
   set style line 5 lt 1 lw 4 lc rgb '#00aaaa' pt -1
   set style line 6 lt 1 lw 4 lc rgb '#cc0000' pt -1
   set grid xtics lc rgb '#555555' lw 1 lt 0
   set grid ytics lc rgb '#555555' lw 1 lt 0
   set xlabel '$t$, c'
   set ylabel '$P$, кН'
   set key bottom right   
   plot 'results/2319.log' using 1:3 with lines ls 2 ti '$P_2$', \
	'results/2319.log' using 1:9 with lines ls 4 ti '$P_9$', \
	'results/2319.log' using 1:19 with lines ls 5 ti '$P_{19}$',\
	'results/2319.log' using 1:29 with lines ls 1 ti '$P_{29}$',\
	'results/2319.log' using 1:49 with lines ls 3 ti '$P_{49}$', \
	'results/2319.log' using 1:54 with lines ls 6 ti '$P_{53}$'	
  \end{gnuplot}
  \caption{Продольные силы в различных сечениях поезда}
 \end{figure} 


\end{document}
